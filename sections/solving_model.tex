\section{Solving models}

\subsection{Primary considerations}

Suppose that we want to control a robot.
We will take two examples of the stepper controlled CNC and a quadrirotor, both apparently having different control models.\newline

Due to its stepper motors, and the supposition that no forces will interfere, the CNC can be controlled in a deterministic way, with no feedback.
The drone, is submitted to the gravity and therefore, is controlled using feedback algorithms, to adjust motors powers in real time.\newline

Beyond those differences, those robots have similarities
\begin{itemize}
    \item [.] They are represented by a state.
    For the CNC, it can be the set of coordinates of its steppers, their speed, acceleration, etc\ldots
    For the drone, it can be, is the set of powers of its motors, and angular positions, speeds, accelerations\ldots
    \item[.] They are solved, ie, an algorithm is used to determine, knowing the current state, the target state.
    For the CNC, it can be a trajectory tracer, for the drone, it can be a set of PIDs, coupled with a trajectory tracer;
    \item[.] They are controllable, ie, the solving algorithm takes parameters that determine its behaviour.
    For the CNC, it can be the acceleration / speed bounds of stepper motors, the number of steps per unit, etc\ldots
    For the drone, it can be PID coefficients.\newline
\end{itemize}

We hereby see that behind those control problems, there are common elements.\newline

An approach, when writing a control algorithm for one of these, can be to neglect those similarities, and build a specific control algorithm.
Another approach that motivates this section, is to evaluate this range of problems in a more abstract way.\newline

The idea here is not to search for similarities in solutions to those problems, but rather to start from an abstract
model of solving system, of which those control problems are particular cases.

%TODO FINISH THE INTRODUCTION, WHY ARE WE BUILDING THIS MATHEMATICA ABSTRACT ?


\subsection{Solving system}

Let $S, C$ be sets, $X : S \times C \rightarrow S$, the triplet $(S, C, X)$ is a solving system;

S is the system's states set.
C is the system's control set.
X is the system's solver.

$S \in S$ is a state.\newline
$c \in C$ is a control element;

A state $s \in S$ can be seen as one representation of a particular system, S being the set of all possible states.
A state can be known, or not, and the aim of a solver will be to compute it.\newline
A control element $c \in C$ can be seen as a set of parameters that will affect the computation of a state, during a solving process.
All of these parameters are know, they do not intend to be determined by the solver, but to control is behaviour


A state $s$ can represent a robot's physical state, (see behind), or the various variables of an equation system,
generally, any set of variable that must be computed;
A control element $c$ can contain acceleration bounds for the trajectory controller of the robot,
or coefficients of the equation system.


\subsection{Solving system}

