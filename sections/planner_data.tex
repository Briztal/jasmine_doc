\newpage

\section{Planner data structures}

This section discusses the format of data structures in the planner implementation.

The planner implementation comprises {\color{red} TODO NUMBER} main data structures :
the trajectory structure, the transition structure, and the planner structure;

\subsection{Trajectory}

A trajectory structure, as introduced in the section about the planner's algorithm, is 
define like below :

\begin{lstlisting}[style=CStyle]

struct trajectory {
	
	//The dimension of the trajectory;
	const size_t t_dim;
	
	//The start index, minimal value of the index;
	const float t_imin;
	
	//The stop index, maximal value of the index;
	const float t_imax;
	
	//An index increment that can cause significant increments on point coordinates;
	const float t_dincr;
	
	//The trajectory function;
	void (*const t_function)(float index, float *dst_array, size_t dst_size);
	
};

\end{lstlisting}

This is the root structure for all trajectories. A particular trajectory will use composition 
to allow pointer cast, like below : 

\begin{lstlisting}[style=CStyle]

struct rdm_trajectory{

	//Composition, with a trajectory struct;
	struct trajectory traj_data;
	
	//rdm_trajectory struct data;
	...	

};

\end{lstlisting}


\subsection{A trajectory's lifetime in the planner}

A trajectory is not created by the planner. An external program is in charge of creating the trajectory of 
the right type, and to transfer it to the planner. 
The planner will store it in a list, process it entirely, providing chosen coordinates.
During its processing, the trajectory can be consulted by another piece of code, for other computation.
When it its processing is complete, the trajectory can be deleted. As the planner did not create it, it 
does not delete it. This task is delegated to another piece of code. 

This transfer of the trajectory to the planner can by done directly by reference : the algorithm creates an 
instance of appropriate type of trajectory, and passes a reference to the planner. 

This solution raises a problem, when applied to a real use case. Indeed, the trajectory itself
is only a part of what will define the movement. Once the current point of a trajectory and its index
have been chosen by the planner, some other piece of code may use this information to complete the target.
For example, if we want to draw a line on a 2D machine, and to increase the speed linearly from one speed to 
another, the speed target must be evaluated after the target trajectory position is determined, from the
chosen index.

In reality, the trajectory is a part of some more complex object that is meant to determine the movement.
This kind of struct cannot contain a trajectory struct directly (composition) because the trajectory struct
only has a meaning when it is composed to form an actual trajectory. 


\subsection{Planner element}

To solve this problem, we will define the planner object structure : 

\begin{lstlisting}[style=CStyle]  

struct planner_elmt {
	
	//A list head, for list storage;
	struct list_head pe_head;
	
	//The trajectory that the planner should process;
	struct trajectory *pe_traj;
	
};

\end{lstlisting}


\subsection{Transition}

A transition references a point where some brutal direction change occurs. Its is formated like below : 

\begin{lstlisting}[style=CStyle]

struct planner_elmt {


};

\end{lstlisting}



\subsection{Transition lifecycle in the planner}

In opposition to the trajectory, the transition is created, and so, deleted by the planner.
When it re




